\documentclass[17pt]{beamer}
\usetheme{CambridgeUS}
\usepackage[utf8]{inputenc}
\usepackage[english]{babel}
\usepackage{amsmath}
\usepackage{amsfonts}
\usepackage{amssymb}
\usepackage{graphicx}
\usepackage{listings}
\author{Viačeslav Pozdniakov}
\title{Introduction to Coq}
\setbeamercovered{transparent} 
\setbeamertemplate{navigation symbols}{} 
\usefonttheme[onlymath]{serif}
%\logo{} 
%\institute{} 
\date{2017-12-07} 
%\subject{} 
\begin{document}

\begin{frame}
\titlepage
\end{frame}

\begin{frame}
\tableofcontents
\end{frame}

\section{Constructive Mathematics}

\begin{frame}{Two types of Mathematics}

\begin{itemize}
  \item Nonconstructive
  \item Constructive
\end{itemize}

\end{frame}

\begin{frame}{Nonconstructive Mathematics}

\emph{Theorem}: There exist positive, irrational numbers $a$ and $b$ such that $a^b$ is rational.

\emph{Proof}:

\begin{enumerate}
  \item If $\sqrt{2}^{\sqrt{2}}$ is rational, use $a = b = \sqrt{2}$.
  \item If $\sqrt{2}^{\sqrt{2}}$ is irrational, use $a = \sqrt{2}^{\sqrt{2}}$ and $b = \sqrt{2}$ then $a^b = 2$.
\end{enumerate}

\end{frame}

\begin{frame}{Propositional logic}

\begin{center}
Disjunction elimination:
$\frac{P\to Q,R\to Q,P\lor R}{\therefore Q}$
\end{center}
\end{frame}

\begin{frame}{William Shakespeare}
\begin{center}
\emph{To be, or not to be}
\end{center}
\end{frame}

\section{Intuitionistic logic}

\begin{frame}{Modus ponens}
\begin{center}
Implication: $P \to Q,\; P\;\; \vdash\;\; Q$
\end{center}
\end{frame}

\begin{frame}{Rules}
\begin{itemize}
	\item \emph{Bottom (absurdity)} $\bot$
	\item $\neg \phi$ is an abbreviation for $(\phi \to \bot)$
	\item $\bot \to \top$
\end{itemize}
\end{frame}

\begin{frame}{Brouwer–Heyting–Kolmogorov interpretation}
\begin{itemize}
	\item A proof of $P\wedge Q$ is a pair $(a,b)$ where $a$ is a proof of $P$ and $b$ is a proof of $Q$.
	\item A proof of $P\vee Q$ is a pair $(a,b)$ where $a$ is $0$ and $b$ is a proof of $P$, or $a$ is $1$ and $b$ is a proof of $Q$.
	\item ...
\end{itemize}
\end{frame}

\section{Curry–Howard isomorphism}

\begin{frame}{Simplified}

Types correspond to propositions.

Values correspond to proofs.

\end{frame}

\begin{frame}[fragile]{Arbitrary Haskell signature}

\lstset{language=Haskell,
                basicstyle=\ttfamily,
                keywordstyle=\color{blue}\ttfamily}
\begin{lstlisting}
foo :: A -> B
\end{lstlisting}

\end{frame}

\begin{frame}{Not only types and values}
\begin{table}
\begin{tabular}{ l | r }
Logic & Programming \\
\hline \hline
implication 	& function type \\
conjunction 	& product type \\
disjunction 	& sum type \\
true formula 	& unit type \\
false formula 	& bottom type \\
\end{tabular}
\end{table}
\end{frame}

\section{Coq}

\begin{frame}{CoC}
Calculus of Constructions extends Curry–Howard isomorphism
\end{frame}

\begin{frame}{Lambda Cube}
???
\end{frame}

\begin{frame}{What is Coq?}
\begin{center}
Proof Assistant
\end{center}
\end{frame}

\section{Coq demo}

\begin{frame}
	\begin{enumerate}
		\item Define natural numbers
		\item Define addition
		\item Prove associativity of addition
		\item Prove commutativity of addition 
	\end{enumerate}
\end{frame}

\end{document}